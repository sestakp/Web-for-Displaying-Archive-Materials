\chapter{Úvod}
Lidstvo již ušlo velmi dlouhou a nelehkou cestu, na které čelilo nejedné náročné překážce. To, že jsme stále tu, značí, že se nám je všechny povedlo překonat. Nyní závisí na nás, zda se budou opakovat stejné chyby a nebo dojde k~posunu a~dělaní chyb vlastních. Jak pravil George Santayana: \textit{„Ti, kteří si nepamatují minulost, jsou odsouzeni k~tomu, aby si ji zopakovali.\footnote{https://citaty.net/citaty/24176-george-santayana-ti-kteri-si-nepamatuji-minulost-jsou-odsouzeni-k/}“} V~dnešní době digitalizace je možné se podrobnosti ze života našich předků dozvědět z~pohodlí našich domovů. Je tedy na nás informaticích poskytnout tuto službu co nejkvalitněji a~umožnit tak nejen odborné veřejnosti, ale i~začínajícímu badateli nahlédnout do historie. Digitalizování archivního materiálu má také pozitivní dopad na jejich životnost a~přístup k~digitalizovaným kopiím prodlužuje životnost originálů. Bohužel v~rámci České republiky doposud neexistuje centralizované řešení a~každý archiv poskytuje pouze své zdigitalizované materiály, což komplikuje vyhledávání a nutí uživatele přecházet mezi více uživatelskými rozhraními. Cílem této práce je tedy sjednotit nashromážděná data a poskytnout je uživateli skrze reaktivní webové rozhraní. 
\newpara
Diplomová práce se skládá z~dvanácti kapitol popisujících kompletní vývojový cyklus tvořeného systému. První dvě kapitoly seznámí čtenáře s~jednotlivými typy archivního materiálu a~systémy zpřístupňujícími archiválie v~rámci České republiky. Čtvrtá kapitola je věnována technologiím optimalizujícím načítání velkých snímků. Kapitola pět popisuje dostupné architektury a~webové technologie pro tvorbu systémů. Šestá kapitola je zaměřena na analýzu datových sad, které jsou výstupem dostupných scrapovacích systémů. Následující dvě kapitoly se věnují popisu požadavků, vymezení funkcionality a~návrhu architektury pro nově vznikající systém. Ze~životního cyklu software vyplývá, že následuje implementace s testováním, které by mělo být součástí vývoje libovolného softwarového díla. Na projektu se po konzultaci s~vedoucím práce podílel i~Ota Malík \cite{Malik}, a~to na implementaci serverové části aplikace, rozšiřovaní scrapovacích modulů a~implementaci alternativního uživatelského rozhraní. Poslední dvě kapitoly této práce se věnují nasazení systému na školní infrastruktuře a~shrnutí výsledků celé práce.