\chapter{Archiválie}
„\textit{Za archiválii \cite{ArchivnictvíSkripta} považujeme psané, obrazové a~zvukové památky dokumentární povahy, vzniklé ze soustavné organické činnosti svých původců, kteří je již ve svých registraturách nepotřebují, avšak byly pro svou společenskou, politickou či vědeckou důležitost vybrány a~určeny k~trvalé úschově}“.

\section{Berní rula}
Berní rula \cite{SestavteSiRodokmen, berniRula} je nejstarší katastr na historickém území Čech. Název je odvozen od staročeského slova berně, které znamená daň. Jedná se o soupis hospodářů na poddanských usedlostech, jejich pozemkové držby a~seznam chovaných hospodářských zvířat. Berní rula vznikla v~polovině 17. století a~měla za cíl co nejdetailněji zmapovat skutečný stav půdy a~vyměřit daňové povinnosti pro jednotlivé usedlosti. Výsledky se využívaly k~odhadu výši výběru do státní pokladny. Berní rula je rozdělena po panstvích, kde je ke každému panství uveden soupis poddaných obcí. Jako jednotka se používal jeden „osedlý“, což byla osoba trvale žijící na daném území. Jednotliví hospodáři jsou v ní seřazeni podle velikosti pozemků. 
U každého pole se kromě celkové výměry evidoval i údaj o tom, jaká část pole byla skutečně oseta.

\section{Lánové rejstříky}
Lánové rejstříky \cite{SestavteSiRodokmen, lanoveRejstriky} obsahují soupis všech hospodářů na území Moravy a~jedná se o jistý ekvivalent berní ruly pro Moravu. Lánové rejstříky byly vydány na začátku 2. poloviny 17. století. Standardně se zde uváděla výměra osévané půdy. Lánové rejstříky byly psány tehdejším úředním jazykem, němčinou. Rozloha pole se uváděla ve starých rakouských plošných mírách, Metz a~Achtel. Metz neboli měřice činila 0,19 hektaru a~achtl byl potom $\frac{1}{8}$ měřice. Obdobně jako v~berní rule jsou seznamy seřazeny podle velikosti vlastněné půdy.

\section{Matriky}
Matriky \cite{SestavteSiRodokmen, digitalizacevArchivech} jsou nejdůležitějším zdrojem informací pro genealogy, jelikož v nich nalezneme seznamy narození, křtů, svateb, úmrtí a~pohřbů. Slovo matrika pochází z~latinského \textit{matricula}, výrazu označujícího seznamy duchovních. Nejstarší dochované matriky jsou evangelické a~pochází až z~poloviny 16. století. V~roce 1784 se matriční knihy staly úředním dokumentem a~byla přesně stanovena jejich struktura, tzn. jak mají být vedeny. V~19. století začaly vznikat i~civilní matriky určené pro nevěřící a~civilní sňatky. V~roce 1949 bylo rozhodnuto, že matriky přebere stát a~spravovat je budou národní výbory. V~archivech jsou pro badatele zpřístupněny pouze neživé matriky, to jsou takové, kde uběhlo 100 let od posledního zapsaného narození nebo 75 let od posledního sezdání nebo úmrtí.

\section{Rektifikační akta}
Rektifikační akta \cite{SestavteSiRodokmen, rektifikacniAkta} vznikla za účelem revize stávajících lánových rejstříků, které obsahovaly mnoho chyb. Rektifikační akta zahrnovala poddanskou i vrchnostenskou půdu pro účely zdanění. U jednotlivých hospodářů jsou uvedeny informace o tom, jaké pozemky drží a~jakou mají rozlohu. Pro Českou berní rulu vznikla obdobná písemnost s~názvem tereziánský katastr.

\section{Pozemkové knihy}
Pozemkové knihy \cite{SestavteSiRodokmen, hledaniPredku} jsou také známé pod názvem gruntovnice nebo gruntovní knihy. Jedná se o ekvivalent dnešního katastru nemovitostí a~najdeme v~nich informace o~držení a~přepisu majetku. Pozemkové knihy byly vedeny v~rámci jednotlivých panství a~velkostatků. Dnes je nalezneme ve státních oblastních archivech. V~rámci archivů jsou většinou uloženy v~samostatných fondech, případně ve fondech panství nebo velkostatků. Tyto knihy se dochovaly z~různých období a~ne vždy na sebe nutně navazují. Některé pozemkové knihy pocházejí již z~16. století, na~jiných panstvích se dochovaly až od počátku 19. století. Důvodem pro chybějící knihy nejspíše ale nebyl nedbalý přístup vrchnostenské správy, jelikož bylo v~jejím zájmu tyto záznamy uchovávat. Používaly se k~evidenci poplatků, které plynuly z~převodu majetku. Častou příčinu jejich nedochování lze spatřovat např. ve změnách vrchnosti daného panství, která si pozemkovou knihu vzala s~sebou při stěhování. Jazyk, kterým byly knihy vedeny, se odvíjel od úředního jazyka dané vrchnosti a~podle oblasti, kde byla kniha vedena. Na našem území se jedná převážně o češtinu a~němčinu. Pozemkové knihy se mohly vést vložkovým systémem nebo chronologicky. Ve vložkovém systému byl pro každý pozemek vyčleněn určitý počet stran, kam se dané změny zapisovaly. Chronologický systém uchovává změny v pořadí, jak se chronologicky udály za sebou, a pro badatele je složitější na orientaci. Jelikož jsou pozemkové knihy častokrát starší než matriky, tak mohou sloužit pro genealogické účely z~období a~lokalit, které nejsou zmapovány žádnou z~dostupných matrik. Zápisy v~pozemkových knihách slouží i~ke zmapování movitého vybavení, jež je v~rámci každého přepisu uvedeno. Kromě věcných informací se zde lze dočíst i~takové zajímavosti, že pozemky bývaly v~historii často přepisovány na nejmladšího syna, čímž je nabourávána tradiční představa o~tom, že dědí vždy nejstarší syn.

\section{Sčítací operáty}
První sčítání lidu \cite{SestavteSiRodokmen, scitaciOperaty} neboli \textit{census}, bylo na českém území provedeno v~roce 1754 na základě patentu vydaného Marií Terezií v~předchozím roce. Bohužel výstupy těchto sčítání byly nepřesné, jelikož obsahovaly agregovaná data bez jmen poddaných. Změna ve struktuře formuláře pro sčítání lidu přišla až o~více než sto let později a~již se více podobala těm, které známe dnes. I~přes modernější strukturu původních formulářů zůstala ale metoda sběru a~zpracování informací stejná, a~proto byli zaznamenáni jen domácí obyvatelé, a~ne skutečné počty přítomných osob. Například v~Praze se podle sčítání nacházelo 68 tisíc lidí, avšak ve skutečnosti zde žilo okolo 140 tisíc osob. V~roce 1869 přišel říšský zákon, který nastavil periodu pro sčítání lidu na 10 let. Seznam sbíraných údajů již obsahoval, zda osoba je přítomna v~místě sčítání, pohlaví, věk, státní příslušnost, rodinný stav, náboženské vyznání a~obcovací řeč. V~roce 1880 přibyly otázky ohledně gramotnosti a~v roce 1890 se začal zjišťovat i~majitel domu. Jedním z~cílů sčítání lidu bylo zjištění ekonomického postavení obyvatel, proto bylo v~rámci sčítání rozlišováno, zda člověk je výdělečně činný či nikoliv a~ve kterém odvětví pracuje. Sčítací operáty jsou v~rámci archivů uloženy v~separátních fondech. Mohou sloužit k~ověření informací z~matriky, případně i k nalezení informací z~míst a~období, kde žádná matrika není k~dispozici.


\section{Soupis poddaných dle víry}
Soupis poddaných dle víry \cite{SoupisPoddanychDleViry} vznikl na popud patentu vydaného roku 1650, který nařizoval vrchnosti v~Čechách pořídit soupis poddaných na základě příslušnosti ke katolické církvi. Na vypracování těchto seznamů měla vrchnost šest týdnů. Každý soupis začínal majitelem panství, případně jeho správcem, a~následoval seznam poddaných seřazených podle obcí. V~seznamu nemuselo být uvedeno duchovenstvo a~vojáci. Předdefinovaná struktura obsahovala jméno, stav, povolání, věk a~údaj o~náboženském vyznání. Základní jednotkou soupisu byla rodina, u~jejíchž členů se zapisoval vztah vůči hospodáři. Na konec soupisu se připojovaly informace o~stavu kostelů a~farních budov. Toto nařízení trvalo jen velmi krátkou dobu, jelikož zanedlouho stačilo vytvořit pouze soupis nekatolíků, a~proto seznam pro Čechy není kompletní a~pro Moravu a~Slezsko není vypracován vůbec.

\section{Urbáře}
Urbáře \cite{SestavteSiRodokmen, urbare} obsahují informace o vlastnictví pozemků, daňové a~pracovní zatížení našich předků. Urbáře jsou obdobně jako berní rula členěny podle panství a~záviselo na nich, v~jakém stavu se dochovaly, jestli vůbec. Nejstarší urbáře jsou církevní a~pocházejí z~2. poloviny 14. století. V~případě, že se urbáře dochovaly z~daného panství pro delší časové období, tak se dají využít k~analýze vývoje daňového zatížení. V~případě, že pro dané území existují i~pozemkové knihy, tak se informace dají kombinovat. Pro genealogy se nejedná o~tak cenný zdroj informací, jelikož obsahují pouze jméno a~příjmení hospodáře. Urbáře jsou uloženy ve fondech panství a~velkostatků v~oblastních archivech.