\chapter{Závěr}
Cílem této práce bylo vytvořit webový systém pro zobrazování digitalizovaných archiválií z~celé České republiky. Před samotnou implementací bylo nutné analyzovat stávající systémy a~pochopit souvislosti v~řešené doméně. U~analýzy stávajících systémů byla snaha identifikovat jejich silné, ale i~stinné stránky. Systém vychází ze stávajících archivních systémů, aby se badatelé nemuseli seznamovat s~novým uživatelským rozhraním. Zároveň byla snaha vyhnout se chybám, které byly v~aktuálních systémech nalezeny ve fázi analýzy stávajícího řešení. Z~analýzy dosavadních systému vzešla myšlenková mapa s~funkcionalitami, jež byly následně implementovány.
\newpara
Před samotným návrhem byla provedena kontrola výstupů u čtyř scrapovacích systémů, které psali předešlí absolventi Fakulty informačních technologií VUT v~Brně. Výstupy dvou systémů byly spárovány s RÚIAN identifikátory a~jeden obsahoval adresy na naskenované folia archiválií. Bohužel bylo zjištěno, že rozhraní některých archivů se změnilo a~odkazy na folia nebyly použitelné. Porovnání těchto scrapovacích systémů pomohlo s~pochopením jednotlivých entit a~vztahů mezi nimi. Z~porovnání jednotlivých scrapovacích systémů byl vybrán a~integrován systém od Jana Valuška. Scrapovací systém byl rozšířen o~vyhledávání obcí včetně souřadnic a~odesílání dat na serverovou část aplikace.
\newpara
 Tvorba detailního návrhu splnila svůj účel a~fáze implementace byla přímočará, a~to díky zpracovanému návrhu.
\newpara
Aktuálně je systém nasazen v~rámci školní infrastruktury a~je veřejně dostupný. V~systému se aktuálně nachází archivní materiál z~osmi archivů. Mezi archivy patří Zemský archiv v~Opavě, Státní oblastní archiv v~Praze, Moravský zemský archiv, Státní oblastní archiv v~Plzni, Archiv hlavního města Prahy, Státní oblastní archiv v~Litoměřicích, Státní oblastní archiv v~Třeboni a~Státní oblastní archiv v~Hradci Králové. V~systému je aktuálně přes 149 000 archivních záznamů, mezi nimiž může uživatel vyhledávat. Jedná se o~lánové rejstříky, matriky, rektifikační akta, pozemkové knihy, sčítací operáty a~urbáře. Bohužel berní rula a~soupis poddaných dle víry doposud vyšly pouze knižně, takže v~systému nejsou zahrnuty. Kromě státního oblastního archivu v Třeboni bylo pro výše zmíněné archivy zprovozněno prohlížení digitalizovaných skenů archiválií.
\newpage
\noindent
Vývoj systému doprovázelo několik problémů, které bylo potřeba řešit. V~rané fázi vývoje byl kladen požadavek na použití čistě relační databáze. Byla snaha tvořit indexy ručně v~relační databázi a~následně podle nich vyhledávat. Vlastní implementace indexování v~rámci relační databáze byla asi 10$\times$ pomalejší při benchmarku než při vyhledávání v ElasticSearch. Dále je potřeba zmínit, že vlastní implementace indexování nebyla technicky tak propracovaná jako osvědčené řešení nabízené databází ElasticSearch. 
\newpara
Samotné scrapování dat a~dotazování na externí úložiště dat vytvořilo v~průběhu implementace mnoho výzev. Funkční zobrazování naskenovaných folií z~externích zdrojů vyžaduje vlastní proxy server a~alternativní konfigurace pro jednotlivé formáty v~rámci \\OpenSeaDragon knihovny.
\newpara
Součástí scrapování dat je doplnění zeměpisných souřadnic pro lokality. Při získávání zeměpisných souřadnic se vyskytlo více problémů a~tato funkcionalita zahrnuje použití geolokačního API a~shlukovacího algoritmu. Prvním zdrojem nepřesností jsou samotná data archivů, kde je často v~rámci názvu obce přiložen komentář. Tento problém je částečně řešen filtrováním frekventovaných klíčových slov. Druhým zdrojem nepřesností je geolokační API, jelikož OpenStreetMap ve svých podmínkách zakazuje přidávání již zaniklých obcí a~použití jiných geolokačních API je zpoplatněno. Posledním zdrojem potencionální nepřesnosti je samotný shlukovací algoritmus. V~případě, že archiválie obsahuje více obcí, tak výsledky jsou korektní. Nejhorší případ pro navržený algoritmus je takový, kdy archiválie obsahuje pouze jednu obec a~existuje více stejnojmenných obcí v~rámci stejného okresu. V~tomto případě algoritmus zvolí obec, která je vzdáleností blíže archivnímu městu.
\newpara
Tvorba uživatelských poznámek vyžadovala před samotný prohlížeč archiválií vykreslit plátno, které zvládá kreslení, interakci s~prvky a~jejich přemisťování. Pro implementaci této funkcionality byla zvolena knihovna Fabric.JS. Problém nastal při integraci \\s~OpenSeaDragon, kde bylo nutné přepsat stávající knihovnu a~publikovat vlastní verzi skrze npm.
\newpara
Obdobně jako každý softwarový produkt i~tento by se dal dále rozšiřovat. V~první řadě by se dalo rozšířit portfolio podporovaných archivů. Dále by bylo vhodné začít ukládat mediální data na lokální server, což by vedlo ke sjednocení formátu a~případné další optimalizace. Z~důvodu, že systém spoléhá na uložení dat v~rámci jednotlivých archivů, tak tyto optimalizace nejsou možné a~případná změna v~archivním systému povede k nutnému zásahu do příslušného scraperu. Anotování archiválií by se dalo rozšířit o~kooperativní režim, kde by uživatelé mohli živě spolupracovat na tvorbě poznámek pro konkrétní folio.