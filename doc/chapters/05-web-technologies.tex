\chapter{Webové technologie}
V této části diplomové práce bude provedeno porovnání běžně dostupných technologií pro tvorbu webových aplikací a~budou představeny základní návrhové vzory, které se při tvorbě webových aplikací využívají. 
\newpara
Nejjednodušší architektura \cite{architekturyEbook} při stavbě webové aplikace je monolitická, kdy je aplikace vyvíjena a~nasazena jako jedna aplikace (Tier), která může komunikovat s~externími službami, například s databází. Při použití monolitické architektury se často využívá návrhový vzor MVC, v němž model uchovává stav aplikace a~je zodpovědný za jeho aktualizaci a~informování view o změně. View je oddělený od logiky aplikace a~je zodpovědný za prezentaci dat uživateli. Controller má za úkol zpracování uživatelských vstupů a~propojení modelu a~view se zachováním nezávislosti mezi nimi. 
\newpara
Systematičtější přístup při vývoji monolitické architektury je strukturalizace aplikace do logických vrstev (layers), kde se standardně používají tři vrstvy. Nejnižší vrstva zapouzdřuje práci s~databází, druhá vrstva se využívá pro implementaci business logiky a~třetí slouží pro prezentaci dat uživateli. 
\newpara
Monolitická architektura má hlavní výhodu v~jednoduchosti a~rychlosti vývoje, ovšem problém zde nastává v~případě, kdy je potřeba aplikaci škálovat na vyšší zátěž. V tomto případě jsme převážně odkázáni na vertikální škálování. 
\newpara
Modernější přístup k~vývoji webových aplikací je plně separovat klientskou část, jež je zodpovědná za prezentaci a~aplikační logiku. Klientská část pak typicky bývá napsána v~reaktivním JavaScriptovém frameworku a~komunikuje se serverovou částí aplikace pomocí REST nebo GraphQL API. Serverová část zůstává třívrstvá, ale s~tím rozdílem, že místo renderování view se vrací pouze data, a to prostřednictvím aplikačního rozhraní většinou ve formátu JSON nebo u starších systémů ve formátu XML.
\newpara
V případě návrhu rozsáhlých aplikací, kde se očekává vysoká zátěž, lze využít návrh rozložení aplikace do jednotlivých mikroslužeb. Monolitická serverová část je rozdělena do více menších aplikací, kde každá aplikace může mít vlastní databázi a~komunikovat s ostatními synchronně pomocí RPC nebo asynchronně pomocí message brokera, jako je například RabbitMQ. Tato architektura je absolutně nejvolnější vzhledem ke škálování aplikace a~umožňuje kromě vertikálního i~horizontální škálování.

\section{Serverové technologie}
Serverové technologie slouží k~tvorbě API, případně monolitických systémů. Víceméně každý modernější programovací jazyk má vlastní aplikační rámec pro tvorbu webových aplikací. Tyto aplikační rámce často sdílí podobné rysy a~přechod na jiný aplikační rámec je pro zkušeného programátora velmi intuitivní a~týká se pouze lehkých změn v~syntaxi daného jazyka a~jmenných konvencí, které se historicky v~daném jazyce využívají. Jednotlivé programovací jazyky se liší v~typování, kde jsou jazyky, co vyžadují statické typování, jiné zase datový typ odvozují, případně i~dynamicky přetypovávají. Jako příklad lze uvést, že v~rámci Javy se více využívá deklarativní programování, kdy některé konstrukce, jako například transakce, lze vyjádřit pomocí anotací. Většina těchto jazyků je dnes interpretována, což za cenu pomalejšího běhu dodává například možnosti modifikace kódu za běhu a~přístup k~objektům pomocí reflexe.

\section{Technologie pro ukládání a~poskytování dat}

Většina těchto aplikací vyžaduje perzistentní uložení dat. Značnou část těchto webových aplikací tvoří OLTP a~OLAP systémy. V~rámci OLTP systémů je kladen velký důraz na konzistenci, jelikož modelují nějaký fyzický podsystém. Mezi modelem a~skutečným fyzickým podsystémem musí existovat isomorfismus. K~dosažení těchto cílů je často volena SQL databáze, jelikož plní podmínky ACID. Další možností pro implementaci úložiště pro OLTP systémy jsou NewSQL databáze, které jsou lépe připraveny do distribuovaného prostředí.
\newpara
Dalším typem úložiště je NoSQL databáze, která neklade takový důraz na konzistenci a~používá model BASE. Jednotlivé NoSQL databáze se od sebe velmi liší a~každá z~nich má specifický případ použití, takže je nutné v~době návrhu zvolit vhodnou databázi. NoSQL databáze jsou v~základu úložiště klíč-hodnota, kde klíče je možné ukládat v~hashovacích tabulkách nebo ve stromové struktuře, což ovlivňuje možnost range scanu a~rychlost vyhledávání. V~rámci NoSQL databází, jež se zaměřují na zvládání velkého množství zápisů, se používá datová struktura LSM tree. Jiné databáze se zase hodí pro optimalizované čtení, kde oproti SQL databázím se netrvá na normalizaci dat. Datová redundance je zaváděna záměrně, a to pro zrychlení dotazování.

\section{Klientské technologie}
Mezi klientské technologie se kromě standardního HTML, CSS a~čistého JavaScriptu také řadí technologie používané u monolitických architektur, jako jsou Blazor, Java Server Pages a~Java Server Faces. Nevýhoda tohoto řešení je, že při požadavku může docházet k~přenačtení celého obsahu, což se dá vyřešit asynchronním voláním například pomocí technologie AJAX nebo v~případě Blazoru může existovat spojení pomocí websocketů.
\newpara
Při použití separátní aplikace pro klientskou část aplikace se v~dnešní době standardně využívají reaktivní JavaScriptové aplikační rámce, které byly navrženy na tvorbu interaktivních webových aplikací, kde se veškeré volání provádí asynchronně, takže neblokuje aplikaci. Standardně se zde využívá vývoj pomocí komponent, kde každá komponenta může mít vnitřní stav. Ve chvíli kdy dojde ke změně vnitřního stavu, dojde následně i~k~překreslení výsledného pohledu. Mezi zástupce těchto technologií patří Vue.JS, Svelte, React nebo Angular. 
